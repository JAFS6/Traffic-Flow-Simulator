	\paragraph{}
	En este capítulo se mostrarán los objetivos que han sido marcados para la realización de este proyecto.

\section{Objetivo principal del proyecto}

	\paragraph{}
	El objetivo principal de este proyecto es la obtención de una aplicación multiplataforma de código abierto que permita observar, de forma gráfica en tres dimensiones, el flujo de tráfico de vehículos en distintas configuraciones de red viaria, que el usuario podrá cambiar mediante la interacción con la aplicación.
	
\section{Red viaria}

	\paragraph{}
	La red viaria con la que trabajará el simulador será una red sencilla, en la cual, habrá puntos por los que entrarán y saldrán los vehículos hacia y desde la porción de red viaria que se esté simulando, habrá intersecciones de mínimo tres vías y máximo cuatro vías y algún mecanismo que nos permita definir curvas sin realizar intersecciones.

\section{Señales de tráfico y peatones}

	\paragraph{}
	
	
\section{Vehículos}
	
	\paragraph{}
	Los vehículos que intervendrán en la simulación serán de al menos tres tipos, coches, autobuses, y camiones no articulados.
	
	\paragraph{}
	Los modelos de estos vehículos serán de poco nivel de detalle con el fin de poder incluir en la simulación tantos como se pueda sin que se vea afectado el rendimiento del sistema.
	
	\paragraph{}
	El usuario podrá cambiar el número de vehículos involucrados en la simulación mediante la interacción con la interfaz gráfica.
	
\section{Conductores}

	\paragraph{}
	Para simular el flujo de tráfico de una forma más realista, el sistema conductista contará con al menos tres tipos de conductores, aquellos que cumplen las normas de circulación siempre, aquellos que las cumplen normalmente y por último, aquellos que casi nunca las cumplen, en menor proporción respecto a los otros dos tipos.
	
\section{Entorno}
	
	\paragraph{}
	Para obtener una simulación con cierto nivel de detalle se realizará un entorno de las redes viarias con algunos edificios, árboles, señales, semáforos, rotondas, pasos de peatones y paradas de autobús.
	
	\paragraph{}
	Así mismo contará con un ciclo diario en el que se podrá cambiar la velocidad del paso del tiempo. Las distintas velocidades serán: 1x, 5x, 15x, 30x, 60x, 120x.
	
\section{Uso de redes viarias reales}

	\paragraph{}
	